\chapter{Conclusion}
\section{Applications and Further Usage}
As already noted in \ref{sec:Motivation} there are multiple appliances in the field of robotics and the gaming industry. The exact use is open due to the generalised representation of the problem as a simple riddle to be solved. Thus every practical problem that can be expressed as a geometric riddle in theoretically $n$-dimensional space can be solved with these algorithms.\\
Also there is the possibility to further improve the algorithms. One could allow non-convex moving objects by, for example, representing them as a set of convex objects. Or for algorithm $fList$ with function list representing the rotation not as a set of planes, but as a continuous function.  This would then give the possibility to check for a collision while rotating and as such removing the need to rotate on a set grid of angles.\\
Another more general idea would be a try on implementing the algorithm in a way such that parallel computation is possible. At the moment this would only be possible for each of the $n$ objects movements making the algorithm idealy $n\cdot3\cdot2$ times faster ( 2 translational dimensions and 1 rotational dimension with increasing/decreasing directions ). But as this parallelization depends on the programming language and enviroment used for applying the algorithm, it is hard to tell how much of an increase in speed could be obtained.
\section{Summary}
In the creation of this thesis tree algorithms for solving object displacement problems were built, implemented and tested. One as a simple approach with an easy geometric way to understand the representation of world objects and two using an analytic approach making computation easier but losing the simple method of representation. \\
All algorithms showed an improvement if compared to a solution using a fixed grid to calculate all possible combinations of objects involved in both accuracy and speed. Due to the nature of the problems, rotations still propose a big hurdle which needs to be overcome for access to a faster solution. Also all problems are bound to have convex moveable objects only. This need for convex objects could be cut if concave objects are build out of multiple convex objects that move together.\\
From the current data, the algorithm $fCell$ emerged as the all-around best solution to general purpose riddles. This algorithm can be applied to a variety of very different problems, due to the fact that in their core they solve a collision detection problem combined with pathfinding which is needed in many applications.   