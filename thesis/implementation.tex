\chapter{Implementation}
\label{cha:Implementation}

\section{Algorithms and Functions}
\subsection{Main search loop}
As proposed in \ref{subsec:confspace}, we create the graph while searching. This requires a certain level of interleaving beetween the search algorithm and the way we work with our objects. In this example, the functions working on the objects are $oneStep(...)$ and $isValid(...)$. All of the rest is needed for the search algorithm, in this case a simple A$^\star$.

\begin{lstlisting}
%while target is not in rim
while(~ismember(target,R))
    %select current node from rim
    next = getNodeFromRim();  
    %calculate rimnodes of current node 
    for i=1:length(directions) % find next node for each search direction
        [possible_next,next_collision_set] = oneStep(next,....); 
        if isValid(possible_next,riddle.b,next_collision_set) % check if node is valid and...
            if ~isInRim(possible_next,R) % ...not in the rim
                R=[R;possible_next]; %if so, add it
                collision_set{length(D)+1} = next_collision_set; %set his collision information
                D=[D;D(next_position)+0.1]; %enter distance to predecessor
                P=[P;next]; %enter next as predecessor
                H=[H;heuristic(possible_next,target)]; %calculate heuristic value
                V=[V;0]; %mark as not visited
            else                    %... already in the rim
                pn_position=find(possible_next,R); %search the node in the rim
                if(D(pn_position)>D(next_position)+0.1) % if node in rim is further away as new node...
                    D(pn_position)=D(next_position)+0.1; %... update distance
                    P(pn_position,:)=next; % and chance predecessor
                end %else do nothing
            end
    end  
end
\end{lstlisting}

$isValid(...)$ is a simple function testing the given configuration from $oneStep(...)$ against configuration space. This is done by simply checking if the configuration point is inside any convex hulls stored in the configuration space.\\
The functions oneStep and getRim are now explained in detail for the representation of objects as points list and function list.
\subsection{Implementation with point list}
As described in \ref{subsec::pointlist} an object can be described as a list of ordered corner points. Starting from there the configuration space for each object is the set of convex hulls of its collision points with each other object. A point in the configuration space can then be described as the set of said subsets.\\
These points form the search nodes for the pathfinding algorithm.
\subsubsection{The function oneStep}
$[\textbf{nextNode}, \textbf{newCollSet} ]=oneStep(\textbf{node, direction, collSet,riddle, jump\_over})$ is the function that provides us with the next node in a specific direction and an updated collision set for that node. This new collision set fits to the new point in the configuration space at the position of the next node.\\
Parameters:
\begin{itemize}
\item \textbf{node}: a point in the configurationspace $C$ used as the starting point.
\item \textbf{direction}: the dimension of $C$ to search a new node on. Signed means search backward, unsigned forward. 
\item \textbf{collSet}: $i$ sets of $n-1$ sized sets giving the collisionsets $C_{O_i-O_j}$ per object $O_i$ for all $n$ objects.
\item \textbf{riddle}: the original set of information for the starting riddle. Needed for recalculation of collSets.
\item \textbf{jump\_over}: flag for signaling if next node should be on the rim of current cell ( jump\_over = -1), or in the next cell ( jump\_over = 1).
\end{itemize}
Returns:
\begin{itemize}
\item \textbf{nextNode}: the next point in the configuration space from \textbf{node} along the dimension \textbf{direction}.
\item \textbf{newCollSet}: the new collisionset for the objects at the configuration \textbf{nextNode}.
\end{itemize}
The fucntion can be divided in two different step types: rotation and translation. \\
All rotations are executed by changing the dimension \textbf{direction}  depending on the sign of \textbf{direction} by one step. The size of the step is determined 
by the needed accuracy. The function rotateObject rotates all cornerPoints of the object around its anchor point.\\
\begin{lstlisting}
...
    tempAdd = zeros(1,length(node)); %build mask for step in rotation direction
    tempAdd(abs(direction))=sign(direction)*rotationStep; %roationstep is set depending on the needed accuracy
    nextNode = node + tempAdd; % add mask to old node

    for object=1:length(riddle.o)
        riddle.o{object} = changeOneObject(nextNode((object-1)*3+1:object*3),riddle.o{object});    %change object according to its new rotation
    end
    
    for object=1:length(riddle.o) %for each object 
        temp = riddle.o;
        temp(object) = [];
        newCollSet{object}= getRims(riddle.o{object}.data,temp,... %generate new collision sets
            length(riddle.o{object}.data),riddle.o{object}.mid);
    end
    return; % and return
\end{lstlisting}
On the other side stands the translation step. As our goal is to identify the next cell in the dimension \textbf{direction}, we iterate over all convex hulls stored in \textbf{collSet} for the object that needs changing and extend their borders. A jump over/to the nearest border in a given direction equals a jump into the next / to the border of the cell.\\
First two points are taken from the convex hull (named points) to recalculate the vector. Due to the fact that each object only resideds in a 2 dimensional space, by solving a linear equation the crossing points on the dimension that is NOT our search dimension $direction$ can be obtained, so that the point in this direction can be calculated.
\begin{lstlisting}
...
        %calculate line from points
        offset = points(i,:);
        vector = points(mod(i,length(points))+1,:)-points(i,:);

        %solve for x and y points
        x =  vector\(node((object_pos-1)*3+1:(object_pos-1)*3+2) - offset) ;
...    
    
...        
        %get point on same x,y coordinate
        %check if line is parallel to searching direction
        if(vector(mod(mod(abs(direction),3),2)+1)<0.001)
            continue;
        end

        %get x to move in y direction and otherwise
        if(mod(abs(direction),3)==1)
            if(x(2,2)<0.001)
                continue;
            end
            p = offset + x(2,2)*vector;%get point on same y
        else
            if(x(1,1)<0.001)
                continue;
            end
            p = offset + x(1,1)*vector;%get point on same x
        end
 \end{lstlisting}
Before we calculate this point, we check if we already reached our target cell by trying to get a non-intercepted connection from the current \textbf{node} to the target node taken from \textbf{riddle}. The flag needs to be tchecked for ALL vectors.       
\begin{lstlisting}
    %check if point is in same cell as target
        if(inTargetCell)
            %find out if direct way to target is possible
            temp=(node(1:2) - offset)';
            A=[vector', -node_to_target];
            sol = A\temp;
            
            point_on_line = offset + sol(1)*vector;
            point_on_line = node(1:2)' + sol(2)*node_to_target;
            
            %check if lines intersect (aka way to target is free )
            if(sol(1)>=0 && sol(1)<=1 && sol(2)>=0 && sol(2)<=1)
                inTargetCell = inTargetCell && false;
            end    
        end
\end{lstlisting}
If all checks out, we calculate the vector from \textbf{node} to the crossing point on the border. If the sign of the vector along the dimension \textbf{direction} equals that of \textbf{direction} a distance is calculated and, if lower than the current minimum, stored together with a possible new node \textbf{nextNode}.
This next node is choosen depending on the flag \textbf{jump\_over} to either be directly infront or behind the point on the border.
\begin{lstlisting}
        %vector from node to temp point on line
        node_to_point = p-node((object_pos-1)*3+1:(object_pos-1)*3+2);
        %get distance to those points if direction is ok
        if sign(node_to_point(mod(abs(direction),3)))~= sign(direction)
            d = inf;
        else
            d = norm(p - node((object_pos-1)*3+1:(object_pos-1)*3+2));
        end
        
        %save new minimum and new point on line
        if d < min_dist
            min_dist = min(min_dist,d);
            if (jump_over==1)
            nextNode((object_pos-1)*3+1:(object_pos-1)*3+2) = p + (node_to_point~=0)*0.001;
            else
            nextNode((object_pos-1)*3+1:(object_pos-1)*3+2) = p - (node_to_point~=0)*0.001;
            end
        end
        
...
\end{lstlisting}
Now after this loop has finished, we have found either a new minimum in the searching direction or we are inside the same cell as the target. \\
If we are in the same cell as the target, we just set the main object $M$ to the targets configuration and return.
\begin{lstlisting}
if inTargetCell && object_pos==1
    nextNode(1:3) = riddle.t.mid;
    return
end
\end{lstlisting}
If we found a new minimum, the collisionsets need to be adapted. For that we recalculate the new configuration state \textbf{nextNode} of our objects with the help of $changeOneObject(...)$.
\begin{lstlisting}
   for object=1:length(riddle.o)
        riddle.o{object} = changeOneObject(nextNode((object-1)*3+1:object*3),riddle.o{object});    
    end
\end{lstlisting}
Afterwards we iterate over the newly generated objects and recalculate the collision sets per object with $getRims(...)$.
\begin{lstlisting}
for object=1:length(riddle.o)
        temp = riddle.o;
        temp(object) = [];
        newCollSet{object}= getRims(riddle.o{object}.data,temp,...
            length(riddle.o{object}.data),riddle.o{object}.mid);
end
\end{lstlisting}
This \textbf{newCollSet} is then returned together with \textbf{nextNode}.


\subsubsection{getRims}
$[\textbf{realRims} ]=getRims(\textbf{objectPoints, rims, objectCount, mid})$ calculates the convex hull for one object to all rims/objects for one configuration. This function is used to update the collision set of an object. 
\begin{itemize}
\item \textbf{objectPoints}: the points of the object.
\item \textbf{rims}: all rims/objects to which the convex hull needs to be calculated.
\item \textbf{objectCount}: The number of points the object has ( length of objectPoints ).
\item \textbf{mid}: the configuration of the object
\end{itemize}
Returns:
\begin{itemize}
\item \textbf{realRims}: the set of convex hulls to all rims/objects for the object with the configuration mid
\end{itemize}

The function starts by iterating over all entities in rims. As a rim is merely a fixed object, there is no need for differenciating beetween object and rim in rims.
\begin{lstlisting}
realRims=cell(length(rims),1);
rimcount=1;
%for each rim in rims
for rim=rims    
    rimData = rim{1}.data;
    %split up the rim in convex parts
    convexRim = [];
    start=rimData(1,:); %pick start point
    vecOld = rimData(2,:) - rimData(1,:); %set start vec
    subsetCount = 1;
    subsets{subsetCount}=rimData(1,:);
    ...
\end{lstlisting}
For each entity the change in angle beetween the vectors connecting the points is calculated, and each object is divided into its convex parts. This is done by     searching along the points of the choosen entity and for each point calculating vector from the previous and the pre-previous point to the current point and comparing their angle. If the angle difference is greater zero, the point is starting a new convex subset. \\
These parts are stored in subset. If the entity is convex itself, subset is of the length one, holding all points of the entity in the first position.
\begin{lstlisting}
...


    for  i=2:length(rimData)
        stop=rimData(mod(i,length(rimData))+1,:); %pick end point
        vec=stop-start; %get vector from start to end
        %get angle diffrence to x-axis for each two vectors
        dirStop = sign((vecOld(2)*vec(1) - vecOld(1)*vec(2)))...
            *acos( (vec(1)*vecOld(1) + vec(2)*vecOld(2))/(norm(vec)*norm(vecOld)));

        if dirStop<=0 %smaller means part will be concave
            subsets{subsetCount} = [subsets{subsetCount};rimData(i,:)];
        else %greater zero means new point is starting new part
            subsets{subsetCount} = [subsets{subsetCount};rimData(i,:)];
            subsetCount = subsetCount+1;
            subsets{subsetCount}=[];
            %subsets{subsetCount} = rim{1}(i,:);
        end
        start=rimData(mod(i-1,length(rimData))+1,:);
        vecOld = stop - start;
    end
 
...
\end{lstlisting}
Afterwards the convex hull is calculated for each subset using the minkowski sum.
\begin{lstlisting}
...

    subsetCount=1;
    convexSubsets = cell(length(subsets),1);
    masks = cell(length(subsets),1);
    poly = cell(length(subsets),1);
    for set = subsets %for each set in subsets
        mSet = cell2mat(set);
        for i=1:size(mSet,1) %we take each value
            v=mSet(i,:);
            %and add the points of the object as seen from the origin ( minkowski sum )
                poly{subsetCount} =...
                    [poly{subsetCount};...
                    (ones(objectCount,1)*(v - mid(1:2))...
                    +objectPoints(1:objectCount,:))]; 
        end

        %extract the points for a convex polygon
        if size(mSet,1)>2
            [convexSubsets{subsetCount},masks{subsetCount}] = getConvexPolygon(mSet,poly{subsetCount},objectCount);
        else
            convexSubsets{subsetCount} = poly{subsetCount};
            masks{subsetCount} = ones(length(poly{subsetCount}),1);
        end
        subsetCount = subsetCount + 1;
    end
    
...
\end{lstlisting}
 As we then hold convex hulls of parts of the entitiy, we need to put them back together. To do that a connection beetween each start and end point of a subset is made. By looking at the angles beetween all possible connections of the points from the minkowski sum, the most outer connection will be choosen.
    
\begin{lstlisting}
...

    if(length(subsets)==1)
        convexRim = convexSubsets{1};
    else        
    for i = 1:length(subsets)-1
        %add first point of first subset
        if i==1
            convexRim = convexSubsets{i}(1,:);
        end
        
        stop = subsets{i};
        stop = stop(size(stop,1),:); % take last point of a set
        
        start = subsets{mod(i,length(subsets))+1};
        start = start(1,:); %take first point of next set
        if(stop==start)
            start = subsets{mod(i,length(subsets))+1};
            start = start(2,:); %take second point
        end
        
        connect = start - stop; % create vector beetween those two
        
        %find point to the right of connect from stop which is in a convex
        %subset. 
        
        %create vectors for stop
        pointsStopSet = poly{i};
        pointsStop = pointsStopSet(size(pointsStopSet,1)...
            -objectCount+1:size(pointsStopSet,1),:);
        vectorStop = pointsStop - ones(size(pointsStop,1),1)*stop; 
        dirStop = acos(vectorStop(:,1)./sqrt(sum(vectorStop.^2,2)))...
            - ones(size(pointsStop,1),1)*acos(connect(1)/norm(connect));
        [v,pStop] = min(dirStop); %select the one farest to the right
        
       ...

        pointsStartSet = poly{mod(i,length(subsets))+1};
        pointsStart = pointsStartSet(1:objectCount,:);
        vectorStart = pointsStart - ones(size(pointsStart,1),1)*start;
        dirStart = acos(vectorStart(:,1)./sqrt(sum(vectorStart.^2,2)))...
            - ones(size(pointsStart,1),1)*acos(connect(1)/norm(connect));
        [v,pStart] = min(dirStart); %select the one farest to the right
  
...
\end{lstlisting}
Afterwards some minor adjustments take place to remove unnecessary lines and points from the hull to reduce "ghost planes". 
Then the finished convex hull of the object with the choosen entity is saved. After each entity in rims is checked, the new set of convex hulls is returned.

\subsection{Implementation with function list}
The implementation as function lists \ref{subsec:functionlist} describes the configuration space simply by the objects at their positions. After eacj step, the object that was moved will be updated and a copy of that set is stored as the new ciollision set for the configuration.
\subsubsection{The function oneStep}
$[\textbf{nextNode}, \textbf{collision\_set} ]=oneStep(\textbf{node, direction, cur\_collision\_set,riddle})$ is the function that provides us with the next node in a specific direction and an updated collision set for that node. This new collision set fits to the new point in the configuration space at the position of the next node.\\
Parameters:
\begin{itemize}
\item \textbf{node}: a point in the configurationspace $C$ used as the starting point.
\item \textbf{direction}: the dimension of $C$ to search a new node on. Signed means search backward, unsigned forward. 
\item \textbf{curr\_collision\_set}: $i$ sets of $n-1$ sized sets giving the collisionsets $C_{O_i-O_j}$ per object $O_i$ for all $n$ objects.
\item \textbf{riddle}: the original set of information for the starting riddle. Needed for recalculation of collSets.
\end{itemize}
Returns:
\begin{itemize}
\item \textbf{nextNode}: the next point in the configuration space from \textbf{node} along the dimension \textbf{direction}.
\item \textbf{collision\_set}: the new collisionset for the objects at the configuration \textbf{nextNode}.
\end{itemize}
In comparison to the header of the other $oneStep(...)$ implementation, only the parameter $\textbf{jump\_over}$ is no longer needed.\\
The function can again be divided into two different steps: rotation and translation.\\
The fucntion can be divided in two different step types: rotation and translation. \\
All rotations are executed by changing the dimension \textbf{direction}  depending on the sign of \textbf{direction} by one step. The size of the step is determined 
by the needed accuracy. The function rotateFunc recalculates the function parameters after the rotation. \\
\begin{lstlisting}
...
    tempAdd = zeros(1,length(node)); %generate mask
    tempAdd(abs(direction))=sign(direction)*rotationStep; %rotationStep depends on the needed accuracy
    nextNode = node + tempAdd; %adapt old node
    
    %rotate object to fit new configuration
    curr_collision_set{object_pos} = rotateFunc(nextNode((object_pos-1)*3+1:object_pos*3),curr_collision_set{object_pos}); 
    collision_set = curr_collision_set; % save new collision set
   return; %and return
\end{lstlisting}
On the other side stands the translation step. Due to the fact, that there are no cells, the goal is to move the object in the \textbf{direction} until it collides with either the border or another object. This is done by checking all elements in the \textbf{curr\_collision\_set} for collision depending on the \textbf{direction}. If none could be found the border of the \textbf{riddle} is used.\\
As the object to be moved is defined by \textbf{direction}, all other objects need to be checked for a possible collision. Therefore a loop iterates over all functions in all objects in \textbf{curr\_collision\_set}, checking if the moving object can be moved to the target without collision and calculating the configuration node if a collision exists in \textbf{direction} by calling $moveToFunc(...)$
 \begin{lstlisting}
...
  if(object_number==object_pos) %skip if object is moving object
        continue;
    end
    %pick object
    object = curr_collision_set{object_number};    

    %pick function from object
    for function_number=1:length(object.coeff)
        func=object.coeff{function_number};
        def=object.def{function_number};
...  

...
        %% check if point is in same cell as target
        if(inTargetCell)
            
            %check if x is ok
            inTargetCell = inTargetCell &&...
                sign(node(1)-def(1)) == sign(riddle.t.mid(1) - def(1)) &&... %left border
                sign(node(1)-def(2)) == sign(riddle.t.mid(1) - def(2)) &&... %right border
                (func(1)==0||(sign(node(1)-ext_x) == sign(riddle.t.mid(1) - ext_x))); %quadratic function only
            
            %check if y is ok
            inTargetCell = inTargetCell &&...
                sign(node(2)-max_y) == sign(riddle.t.mid(2) - max_y) &&... %upper border
                sign(node(2)-min_y) == sign(riddle.t.mid(2) - min_y); %lower border
        end
        
        %% get nextNode in search direction. if function not in the way, dist = inf.
        [tempNode,dist] = moveToFunction(node,direction,curr_collision_set{object_pos},riddle.b,func,def,object.above{function_number});
        
        %save new minimum ( closest function in the way ) and nextNode
        if abs(dist) < abs(min_dist)
            min_dist = min(abs(min_dist),abs(dist))*sign(dist);
            nextNode=tempNode;
        end
...
\end{lstlisting}
After that if a node with the closest distance was choosen the collision set is recalculated. If not $moveFunc(...)$ moves the object to the border of \textbf{riddle}.
Also if the object can be moved to the target without collision, the target will be set as the next node and the function returns.
\begin{lstlisting}
...
%% build new collision set from chosen node
curr_collision_set{object_pos}=moveFunc(min_dist,direction,curr_collision_set{object_pos});
collision_set = curr_collision_set;

if inTargetCell && object_pos==1
    nextNode(1:3) = riddle.t.mid;
    return
end
\end{lstlisting}

\subsubsection{moveFunc}
$[\textbf{object}]=moveFunc(\textbf{diff, dir, object})$ recalculates the function parameters stored in \textbf{object}.coeff depending on the direction \textbf{dir} and distance \textbf{diff}  the \textbf{object} needs to be moved.\\
Parameters:
\begin{itemize}
\item \textbf{diff}:the signed distance the object is moved.
\item \textbf{dir}: the dimension of $C$ to search a new node on. Signed means search backward, unsigned forward. 
\item \textbf{object}: the object which needs adaption.
\end{itemize}
Returns:
\begin{itemize}
\item \textbf{object}: the adapted object.
\end{itemize}
The function $moveFunc(...)$ distinguishes beetween movement in $x$- and $y$-direction.\\
 For movement along the$x$-axis the objects functions coefficients are recalculated by substraction of \textbf{diff} from x for every function in the object. After finishing iterating over the functions, the definition range is adapted and the new anchor point is set.
\begin{lstlisting}
...
for function_number=1:length(object.coeff)%iterate over all functions
       func = object.coeff{function_number};
       if func(1)==0 %function is linear
           %b(x-d)+c = bx + ( -bd + c)
           func(3)=-func(2)*diff+func(3);  
       else %function is quadratic
           %a(x-d)^2 + b(x-d) + c =
           %ax^2 - 2adx + ad^2 + bx - bd + c =
           %ax^2 + (-2ad + b)x + (ad^2 - bd + c)
           b = -2*func(1)*diff + func(2);
           c = func(1)*diff^2 - diff*func(2) + func(3);
           func(2)=b;
           func(3)=c;
       end
       object.coeff{function_number}=func; %save new coefficients
       object.def{function_number}=object.def{function_number}+diff;% save new definition range
    end
    object.mid(1) = object.mid(1) + diff;%set new anchor point
...
\end{lstlisting}
Movements alogn the $y$-axis are easier, as they require to only chance the constant coefficient of a function. 
\begin{lstlisting}
...
for function_number=1:length(object.coeff)%iterate over all functions
       func = object.coeff{function_number};
       func(3) = func(3) - diff; %change constant coefficient
       object.coeff{function_number}=func;       
    end
    object.mid(2)=object.mid(2) - diff; %set net anchorpoint
...
\end{lstlisting}

\subsubsection{moveToFunction}
$[\textbf{nextNode},\textbf{dist}]=moveFunc(\textbf{node, direction, object, border, func, def, above})$ checks if the function \textbf{func} defined in the range \textbf{def} lies in the way of the \textbf{object} trying to move in the \textbf{direction} or if the \textbf{border} needs to be used. The function calculates the configuration point \textbf{nextNode} together with the travelled distance \textbf{dist}.
Parameters:
\begin{itemize}
\item \textbf{node}: the current node the movement should be executed from.
\item \textbf{direction}: the dimension of $C$ to search a new node on. Signed means search backward, unsigned forward. 
\item \textbf{object}: the object which wants to be moved.
\item \textbf{border}: the border of the riddle.
\item \textbf{func}: the function which could be laying in the way.
\item \textbf{def}: the definition range of the function.
\item \textbf{above}:wether the object the function \textbf{func} borders lies above or below \textbf{func}
\end{itemize}
Returns:
\begin{itemize}
\item \textbf{nextNode}: the next configuration point after moving the \textbf{object}.
\item \textbf{dist}: the distance the \textbf{object} has been moved.
\end{itemize}
The function $moveFunc(...)$ iterates over all functions in \textbf{object} and checks if there exists a crossing point beetween them and the function \textbf{func}.
This point is then saved together with the distance and the point with the minimal distance is returned.\\
First the values of the function \textbf{func} are calculated and if the movement is along the x-coordinate, compared to those of the \textbf{object} functions. If no common range exists, the function can not collide.
\begin{lstlisting}
...
%calculate funcValue at ends of defenition range
funcValue_min = (def(1)^2)*func(1) + def(1)*func(2) + func(3);
funcValue_max = (def(2)^2)*func(1) + def(2)*func(2) + func(3);

for obj_function_number = 1:length(object.coeff)
    %get current function and range
    obj_func = object.coeff{obj_function_number};
    obj_def = object.def{obj_function_number};
    
 
    if(mod(abs(direction),3)==1)
        %  move along x coordinate (right/left)
        %% calculate function values
        objValue_min = (obj_def(1)^2)*obj_func(1)  + obj_def(1)*obj_func(2) + obj_func(3);
        objValue_max = (obj_def(2)^2)*obj_func(1)  + obj_def(2)*obj_func(2) + obj_func(3);
        
        %% get the y-range used by both function ( inner borders )
        min_y = (objValue_min>funcValue_min)*objValue_min + (objValue_min<=funcValue_min)*funcValue_min;
        max_y = (objValue_max<funcValue_max)*objValue_max + (objValue_max>=funcValue_max)*funcValue_max;
        
        if(min_y > max_y) % if no common range exists, jump to next function
            if(sign(direction)==-1)
                diff_min = border(1,1) - obj_def(1);
                diff_max = border(1,1) - obj_def(2);
            else
                diff_min = border(3,1) - obj_def(2);
                diff_max = border(3,1) - obj_def(1);
            end
\end{lstlisting}
  If a common range exists, the $x$ values at the edges of the overlapping $y$ values are calculated.
\begin{lstlisting}
...
            
            %% calculate x values to min_y and max_y
            %linear functions
            if(func(1)==0) %distinguish beetween linear and quadratic
                if(func(2)==0)
                    funcX_min = def(1);
                    funcX_max = def(2);
                else
                    funcX_min = (min_y - func(3))/func(2);
                    funcX_max = (max_y - func(3))/func(2);
                end
            else%quadratic functions
                [x1_min,x2_min]=solve(poly2sym([func(1),func(2),func(3)-min_y]));
                [x1_max,x2_max]=solve(poly2sym([func(1),func(2),func(3)-max_y]));
                
                %choose x nearest to the object using search direction sign
                if(sign(direction)==-1)
                    funcX_min = (x1_min>x2_min)*x1_min +(x1_min<=x2_min)*x2_min;
                    funcX_max = (x1_max>x2_max)*x1_max +(x1_max<=x2_max)*x2_max;
                else
                    funcX_min = (x1_min<x2_min)*x1_min +(x1_min>=x2_min)*x2_min;
                    funcX_max = (x1_max<x2_max)*x1_max +(x1_max>=x2_max)*x2_max;
                end
            end
...
\end{lstlisting}
The same calcuations are also made for the current \textbf{object} function to calculate $objX_min$ and $objX_max$. The difference beetween the \textbf{object} function and the \textbf{func} values are used to determine the distance the \textbf{object} needs to be moved towards the function \textbf{func}. The minimum distance is then saved together with the new configuration point.
\begin{lstlisting}
...           
            
            diff_min = funcX_min - objX_min; %calculate distances
            diff_max = funcX_max - objX_max;
        end
        
        if(diff_min==0 || diff_max==0) %check if object lies besides function
            if(sign(direction)==sign(above)) %if object is moving away from function
                continue;
            end
        end
        
        %% choose smaller distance...
        if(abs(diff_min)<abs(diff_max)&&abs(diff_min)<abs(dist))
            nextNode(abs(direction)) = node(abs(direction)) + diff_min; %... and nextNode;
            dist = diff_min;
        elseif(abs(diff_max)<abs(dist))
            nextNode(abs(direction)) = node(abs(direction)) + diff_max; %... and nextNode;
            dist = diff_max;
        end
    end
...
\end{lstlisting}
If the movement is along the y-coordinate a simple look at the defenition ranges of the \textbf{object} functions rules out all functions that are not in the way.
If that is the case, the border is taken as the nearest function.
\begin{lstlisting}
...

% move along y coordinate ( up/down)
else
        %% get the x-range used by both function ( inner borders )
        min_y = (def(1)>obj_def(1))*def(1) + (def(1)<=obj_def(1))*obj_def(1);
        max_y = (def(2)<=obj_def(2))*def(2) + (def(2)>obj_def(2))*obj_def(2);
        
        %% check if function lies in the way
        if(min_y > max_y) % if no common range exists, take borders
            %% calculate function values
            objValue_min = (obj_def(1)^2)*obj_func(1) + obj_def(1)*obj_func(2) + obj_func(3);
            objValue_max = (obj_def(2)^2)*obj_func(1) + obj_def(2)*obj_func(2) + obj_func(3);
            
    ...
\end{lstlisting}
If there is a common range, the $objValue$s are calculatet with the common function range and the distance is stored together with the new configuration point.
\begin{lstlisting}
        elseif(min_y<=max_y) %start to find collision
            
            %% calculate function values
            objValue_min = (min_y^2)*obj_func(1) + min_y*obj_func(2) + obj_func(3);
            objValue_max = (max_y^2)*obj_func(1) + max_y*obj_func(2) + obj_func(3);
    ...

    ...
            diff_min = funcValue_min - objValue_min; %calculate distances
            diff_max = funcValue_max - objValue_max;
            
            if(diff_min==0 || diff_max==0) %check if object lies besides function
                if(sign(direction)==sign(above)) %if object is moving away from function
                    continue;
                end
            end
        end

        %% choose smaller distance...
        if(abs(diff_min)<abs(diff_max) && abs(diff_min)<abs(dist))
            nextNode(abs(direction)) = node(abs(direction)) + diff_min; %... and nextNode;
            dist = diff_min;
        elseif(abs(diff_max)<abs(dist))
            nextNode(abs(direction)) = node(abs(direction)) + diff_max; %... and nextNode;
            dist = diff_max;
        end
   
end
\end{lstlisting}
