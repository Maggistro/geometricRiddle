\chapter{Results}
If we apply the finished Algorithm to various test data sets, we can get an impression of what the program is capable of.
For each riddle beeing solved, the algorithms are executed a number of times to calculate an average runtime.\\
In both cases a A$^\star$ was used as the search algorithm on the resulting graph. As a distance measurement the heuristic in form of the 2-norm of the vector beetween current position and target was added to the stepcount with each step weighting 0.1 units. The starting distance to beetween main object and target is
10.1980 units.
\section{Rotation and Translation}
Using small riddles tailored to either solving by rotation or translation only, an estimation 

\begin{figure}[H]
%two graphics with rotation and translation
\end{figure}
Riddle with one rotation only object:
\begin{table}
\centering
\begin{tabular}{l||c|c|c|}
Algorithm& fastest & slowest & medium\\\hline
pList &  18.8819 & 22.8457  & 21.0269 \\
fList  & 0.3523&0.3703  &0.3587 \\
\end{tabular}
\caption{Times in seconds for rotation riddle.}
\end{table}
Riddle with translation only object:
\begin{table}
\centering
\begin{tabular}{l||c|c|c|}
Algorithm& fastest & slowest & medium\\\hline
pList &  0.1211& 0.1272  & 0.1231 \\
fList  & 0.0219 & 0.0241 & 0.0228 \\
\end{tabular}
\caption{Times in seconds for translation riddle.}
\end{table}

\section{Small riddles}
Two simple small riddles with 2 and 4 objects as moveable obstacles.

\begin{figure}[H]
%two graphics with 2 and 4
\end{figure}
Riddle with 2 objects:
\begin{table}
\centering
\begin{tabular}{l||c|c|c|}
Algorithm& fastest & slowest & medium\\\hline
pList & 4.7801& 5.1930&4.9305\\
fList  & 1.0082 & 1.0315& 1.0149\\
\end{tabular}
\caption{Times in seconds for small riddle with 2 obstacles.}
\end{table}
Riddle with 4 objects:
\begin{table}
\centering
\begin{tabular}{l||c|c|c|}
Algorithm& fastest & slowest & medium\\\hline
pList & 85.1726 & 90.9629 & 89.3621\\
fList  & 18.2735 & 21.2764 & 19.6290\\
\end{tabular}
\caption{Times in seconds for small riddle with 4 obstacles.}
\end{table}

\section{Medium riddles}
Bigger riddle with 6 and 8 moveable objects.\\
\begin{figure}[H]
%two graphics with 6 and 8
\end{figure}
Riddle with 6 objects:
\begin{table}
\centering
\begin{tabular}{l||c|c|c|}
Algorithm& fastest & slowest & medium\\\hline
pList &  395.0670 & 415.7986 & 403.9329\\
fList  &  157.0453 &  167.1498 & 164.2913\\
\end{tabular}
\caption{Times in seconds for medium riddle with 6 obstacles.}
\end{table}
Riddle with 8 objects:
\begin{table}
\centering
\begin{tabular}{l||c|c|c|}
Algorithm& fastest & slowest & medium\\\hline
pList & 1790.5 &1972.5 &1855.4\\
fList  & & &\\
\end{tabular}
\caption{Times in seconds for medium riddle with 8 obstacles.}
\end{table}

\section{Interpretation}
As we can see the algorithm working with functions is much faster. But still it scales badly with an increasing amount of objects to work with.
If we plot these two against each other we see a exponential grow depending on the number of objects involved.

